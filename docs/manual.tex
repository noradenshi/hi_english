\documentclass[12pt,a4paper]{article}

% --- Pakiety ---
\usepackage[utf8]{inputenc}
\usepackage[T1]{fontenc}
\usepackage[polish]{babel}
\usepackage{graphicx}
\usepackage{geometry}
\usepackage[hidelinks]{hyperref} % hidelinks usuwa czerwone ramki
\usepackage{titlesec}
\usepackage{float} % Pozwala na użycie [H] do wymuszenia pozycji
\usepackage{subcaption}
\usepackage{xcolor}

% --- Konfiguracja strony ---
\geometry{margin=2.2cm}
\setlength{\parindent}{0pt}
\setlength{\parskip}{1em}

% --- Komenda do wymuszania pozycji obrazów (Tekst -> Obrazy) ---
\newcommand{\imagepair}[5]{
    \begin{figure}[H] % [H] wymusza pozycję "TUTAJ"
        \centering
        \begin{subfigure}{0.48\textwidth}
            \centering
            \includegraphics[width=\textwidth]{screenshots/#1}
            \caption{#2}
        \end{subfigure}
        \hfill
        \begin{subfigure}{0.48\textwidth}
            \centering
            \includegraphics[width=\textwidth]{screenshots/#3}
            \caption{#4}
        \end{subfigure}
        \caption{#5}
    \end{figure}
}

\newcommand{\imagesingle}[3]{
    \begin{figure}[H]
        \centering
        \includegraphics[width=0.45\textwidth]{screenshots/#1}
        \caption{#2}
        \label{fig:#3}
    \end{figure}
}

\title{Manual Projektu \\ \textbf{Hi English}}
\author{Dawid Bar}
\date{Luty 2026}

\begin{document}

\maketitle
\newpage
\tableofcontents
\vspace{2em}
\listoffigures
\newpage

\section{Wstęp i Menu Główne}
Aplikacja \textbf{Hi English} została utworzona aby zachęcić najmłodszych do nauki języka angielskiego. Domyślnym ekranem aplikacji jest ekran ćwiczeń, gdzie użytkownik widzi swój postęp w codziennym wyzwaniu ukończenia 5-ciu zadań, oraz listę dostępnych gier.

\imagesingle{exercises.png}{Główny panel wyboru ćwiczeń}{exercises}

\section{Moduły Gier Edukacyjnych}
Dla użytkownika dostępne są dwie gry. \textit{Matching Game}, którego celem jest dopasowywanie w pary angielskich słów z ich polskimi tłumaczeniami, oraz \textit{Spelling Game}, gdzie należy ułożyć tłumaczenie na angielski polskiego słowa z rozsypanki liter.
Gry nagradzają gracza za każdy etap, oraz za ukończenie całej serii 20-u zadań.

\imagepair{matching_game.png}{Gra w dopasowywanie}{spelling_game.png}{Moduł pisowni}{Interfejsy gier edukacyjnych.}

\section{System Słownika i Wyszukiwania}
Słownik umożliwia użytkownikowi sprawdzenie definicji słów wykorzystywanych w zadaniach. Zapisany jest w lokalnej bazie danych SQLite. Można go na bieżąco przeszukiwać i używać liter na boku do "przeskakiwania" do słów na nią się zaczynających.
Słowa można kliknąć by je podglądnąć.

\imagepair{dictionary.png}{Widok słownika}{dictionary_search.png}{Filtrowanie słówek}{Podstawowe funkcje słownika.}

\section{Dom i Sklep}
Zwierzątko gracza oraz sklep mają za zadanie dać dziecku motywację do dalszej nauki.
Punkty zdobyte w sekcji ćwiczeń mogą wymieniać na nowe przedmioty i zobaczyć swojego pupila w nowym wydaniu.

\imagepair{pet.png}{Widok domu}{store.png}{Sklep z przedmiotami}{Personalizacja.}

\section{Ustawienia}
Aplikacja umożliwia użytkownikowi ustawienie powiadomień, które codziennie o wskazanej porze będą wysyłane jeżeli ten nie wykona wcześniej swojego codziennego zadania. Jeśli użytkownik zablokuje powiadomienia w systemie android, aplikacja wykrywa to przy próbie zmiany ustawień i wyświetla stosowny komunikat.

\imagepair{settings.png}{Statystyki i ustawienia}{settings_notifications_blocked.png}{Alert o braku uprawnień}{Ekran ustawień.}

\section{Panel Root}
Sekcja Root jest chroniona hasłem (domyślnie: root123). Po zalogowaniu się administrator widzi dodatkowe opcje, które są ukryte przed zwykłym użytkownikiem, w tym podgląd stanu preferencji i możliwość edytowania podstawowych z nich.

\imagepair{settings_root.png}{Ekran ustawień Root}{settings_root_extented.png}{Dalsza część ustawień Root}{Dostęp do funkcji specjalnych.}

\subsection{Zarządzanie Słownikiem w Trybie Root}
W trybie Root użytkownik może dowolnie edytować bazę słów. Można dodawać nowe i edytować lub usuwać już istniejące.

\imagepair{dictionary_root_editview.png}{Aktywacja edycji}{dictionary_root_edit.png}{Formularz edycji słowa}{Narzędzia administracyjne słownika.}

\subsection{Widoczność Słownictwa}
Root ma również możliwość ukrywania słówek ze słownika - co również wyłącza je z użycia w grach. System zarządza flagami widoczności bezpośrednio w bazie danych Room.

\imagepair{dictionary_root_hide.png}{Opcja ukrywania słowa}{dictionary_root_hidden.png}{Widok słów ukrytych}{Zarządzanie stanem widoczności elementów.}

\subsection{Zarządzanie przedmiotami}
Kolejną możliwością w tym trybie Root jest sprzedawanie przedmiotów. Zwykły użytkownik nie może pozbyć się raz nabytych rzeczy.

\imagesingle{store_root.png}{Sprzedaż przedmiotów}{Sklep Root'a}

\section{Podsumowanie}
Projekt Hi English jest prostą aplikacją do nauki języka angielskiego. Umożliwia również administratorowi (w domyśle - rodzicowi) na zarządzanie słówkami jakie dziecko dostaje w swoich zadaniach. Tryb administratora również oferuje intuicyjne debugowanie aplikacji. 

\end{document}